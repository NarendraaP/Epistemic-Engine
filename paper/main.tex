\documentclass[journal]{IEEEtran}

% Packages
\usepackage{cite}
\usepackage{amsmath,amssymb,amsfonts}
\usepackage{algorithmic}
\usepackage{graphicx}
\usepackage{textcomp}
\usepackage{xcolor}
\usepackage{hyperref}

% Document metadata
\def\BibTeX{{\rm B\kern-.05em{\sc i\kern-.025em b}\kern-.08em
    T\kern-.1667em\lower.7ex\hbox{E}\kern-.125emX}}

\begin{document}

\title{The Epistemic Engine: Enforcing Constitutional Invariants in Cosmological Visualization}

\author{Narendra~Polisetti
\thanks{N. Polisetti is an independent researcher. Email: narendraa1996@gmail.com}}

\markboth{IEEE Visualization and Computer Graphics}%
{Polisetti: The Epistemic Engine}

\maketitle

\begin{abstract}
Modern astronomical visualizations conflate observational data, theoretical models, and simulated content without explicit provenance labels, creating an ``Epistemic Gap'' that obscures the distinction between what we \emph{see} and what we \emph{infer}. We present the \textbf{Epistemic Engine}, a visualization framework that enforces three constitutional invariants: (I) \textbf{Labeling} -- every data point must have an explicit epistemic status (Observed, Inferred, or Simulated); (II) \textbf{Reference} -- motion must be relative to parent frames, not absolute coordinates; and (III) \textbf{Access} -- users must always know which truth filter is active. We implement this framework using a PostgreSQL provenance database, Python adjudicator pipelines, and OpenSpace with a custom Truth Slider interface. Case studies with Gaia DR3 stellar data (100,000 stars, L0-Observed) and Laniakea supercluster structure (5,000 galaxies, L1-Inferred) demonstrate how epistemic rigor can be maintained across 13 orders of magnitude in spatial scale (10\textsuperscript{8}m to 10\textsuperscript{23}m). Our system rejects unlabeled data at ingestion time and provides explicit UI controls for toggling between truth levels, ensuring that scientific visualizations remain epistemologically honest.
\end{abstract}

\begin{IEEEkeywords}
Scientific visualization, provenance, epistemic labeling, astronomy, cosmology, uncertainty visualization, data quality, OpenSpace
\end{IEEEkeywords}

% Include sections
\section{Introduction}

% TODO: Write introduction section

% Key points to cover:
% - The problem: Epistemic gap in modern astronomy visualizations
% - Example: Hubble images vs. simulations (how do we tell them apart?)
% - Consequences: Public misunderstanding, educational confusion
% - Our solution: Constitutional framework with 3 invariants
% - Contributions of this paper


\section{Related Work}

\subsection{Uncertainty Visualization}

% TODO: Cite Bonneau et al. uncertainty survey
% Discuss how uncertainty ≠ epistemic status

\subsection{Provenance in Visualization}

% TODO: Cite provenance frameworks
% Explain difference between data lineage and truth labeling

\subsection{Astronomical Visualization Systems}

% TODO: OpenSpace, WorldWide Telescope, Uniview
% Critique: None enforce epistemic separation


\section{The Three Epistemic Invariants}

% This is the THEORY section (the core contribution)

\subsection{Invariant I: Labeling (Provenance)}

% Rule: Every data point must have an explicit epistemic status
% Database schema: epistemic_status_type ENUM
% Rejection policy: No unlabeled data allowed

\subsection{Invariant II: Reference (Relative Motion)}

% Rule: Motion must be relative to parent frames
% Scene graph: CMB → Galaxy → Sun → Earth
% No absolute coordinates

\subsection{Invariant III: Access (User Interface)}

% Rule: User must ALWAYS know which truth filter is active
% Implementation: Truth Slider HUD
% Anti-pattern: Seamless blending of real and fake data

\subsection{Formal Definition}

% TODO: Mathematical formalization of invariants
% Consider using set theory or logic notation


\section{System Architecture}

% This is the IMPLEMENTATION section

\subsection{Four-Plane Design}

% Describe the 4 layers:
% 1. Truth Store (PostgreSQL)
% 2. Adjudicator (Python pipelines)
% 3. Export Layer (.speck, .csv, .obj)
% 4. Viewer (OpenSpace + Truth Slider)

\subsection{Truth Store: PostgreSQL Schema}

% Show schema.sql
% Explain epistemic_status_type enum
% JSONB provenance metadata

\subsection{Adjudicator: Python Pipelines}

% Ingestion scripts
% Rejection mechanism for unlabeled data
% Example: ingest_gaia.py

\subsection{Viewer: OpenSpace with Truth Slider}

% Lua scripting
% Keybindings (1/2/3)
% HUD implementation

\subsection{Binary Octree System (Phase 5)}

% LOD framework for massive catalogs
% Parent nodes: brightest 10%
% Scalability analysis


\section{Case Studies}

\subsection{Case Study 1: Gaia DR3 Stellar Catalog}

% Dataset: 100,000 brightest stars
% Epistemic Status: L0 - OBSERVED
% Challenge: Error margins, parallax uncertainties
% Results: Database rejection rate, quality metrics

\subsection{Case Study 2: Laniakea Supercluster}

% Dataset: ~5,000 galaxies, cosmic flows
% Epistemic Status: L1 - INFERRED
% Challenge: Filling the "empty middle" (10^20m scale)
% Results: User study? Clarity of provenance?

\subsection{Case Study 3: CMB Velocity Vector}

% Dataset: Single arrow (369 km/s towards Leo)
% Epistemic Status: L2 - SIMULATED
% Challenge: Representing abstract kinematic data
% Results: Educational value

\subsection{Truth Slider Validation}

% User testing (if available)
% Comparison to traditional "blended" visualizations
% Cognitive load analysis


\section{Conclusion}

\subsection{Contributions}

% Summarize the 3 main contributions:
% 1. Theoretical: The 3 epistemic invariants as a constitutional framework
% 2. System: Four-plane architecture (Truth Store, Adjudicator, Export, Viewer)
% 3. Empirical: Case studies demonstrating enforcement across 13 orders of magnitude

\subsection{Limitations}

% Be honest about what we haven't solved:
% - Full Gaia DR3 (1.8 billion stars) requires C++ renderer
% - No streaming LOD system yet (pre-export only)
% - Limited user testing
% - OpenSpace dependency

\subsection{Future Work}

% - Production C++ point cloud renderer
% - Real-time streaming from PostgreSQL
% - Comparative user study (epistemic vs. blended visualizations)
% - Extension to other domains (medical imaging, climate data)

\subsection{Broader Impact}

% Why this matters:
% - Education: Students deserve to know what's real
% - Public trust in science
% - Visualization ethics

% Final statement: "Show me the universe as it is, not as I wish it to be."



% Bibliography
\bibliographystyle{IEEEtran}
\bibliography{references}

\end{document}
